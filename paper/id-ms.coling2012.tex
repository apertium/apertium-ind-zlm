%% Example of a LaTeX source file for a COLING-2012 submission
%% last updated: July 10, 2012
\documentclass[10pt,a5paper,twoside]{article}
\usepackage{coling2012}
\usepackage{booktabs}
\usepackage{times}
\usepackage{multirow}
\usepackage{alltt}
\usepackage[small,bf]{caption}

\newcommand{\tag}[1]{{\small{\texttt{#1}}}}

\title{Rule-based Machine Translation between Indonesian and Malaysian}

\author{$Raymond~Hendy~Susanto^{1}$\\
{\small  	(1) Department of Computer Science, National University of Singapore\\ 
  \texttt{raymondhs@nus.edu.sg} \\ 
}}

\begin{document}
\maketitle

\abstractEn{
We describe the development of a bidirectional machine translation system between Indonesian and Malaysian, two closely related Austronesian languages widely spoken by 180 million speakers around the world. The system is based on the re-use of free and publicly available resources, such as the Apertium machine translation platform and Wikipedia. We also present our approaches to overcome the data scarcity problems in both languages by exploiting the morphology similarities between the two.}

\keywordsEn{machine translation, Malay languages, morphology}

\newpage
\section{Introduction}

\section{System}
The system is based on the Apertium machine translation platform \citep{apertium/2011}.\footnote{\url{http://www.apertium.org/}} The platform was originally aimed at the Romance languages of the Iberian peninsula, but has also been adapted for other, more distantly related language pairs. The whole platform, both programs and data, are licensed under the Free Software Foundation's General Public Licence\footnote{\url{http://www.fsf.org/licensing/licenses/gpl.html}} (GPL) and all the software and data for the 33 supported language pairs (and the other pairs being worked on) is available for download from the project website.

\subsection{Architecture of the system}

\begin{figure*}[htbp]
\begin{center}
 \includegraphics[width=0.8\textwidth]{architecture.pdf}
\end{center}
\caption{The pipeline architecture of the Apertium system.}
\label{fig:modules}
\end{figure*}

The Apertium translation engine consists of a Unix-style \emph{pipeline} or
\emph{assembly line} with the following modules (see Fig.~\ref{fig:modules}):  
\begin{itemize}
\item A \emph{deformatter} which encapsulates the format information
 in the input as \emph{superblanks} that will then be seen
 as blanks between words by the other modules.
\item A \emph{morphological analyser} which segments the text in
  surface forms (SF) (\emph{words}, or, where detected, multi-word lexical
  units or MWLUs) and for each, delivers one or more \emph{lexical
    forms} (LF) consisting of \emph{lemma}, \emph{lexical category} and
  morphological information. 
\item A \emph{morphological disambiguator} (constraint grammar) which chooses, using linguistic rules
  the most adequate sequence of morphological analyses for an ambiguous sentence. 
\item A \emph{lexical transfer} module which reads each SL LF 
  and delivers the corresponding target-language (TL) LF
  by looking it up in a bilingual dictionary encoded as an FST
  compiled from the corresponding XML file. The lexical transfer module may
  return more than one TL LF for a single SL LF.
\item A \emph{lexical selection} module which chooses, based on context 
  rules the most adequate translation of ambiguous source language LFs.
\item A \emph{structural transfer} module which
    performs local syntactic operations, is compiled from XML files containing rules that 
    associate an \emph{action} to each defined LF \emph{pattern}. Patterns are applied left-to-right, and the 
    longest matching pattern is always selected.
\item A \emph{morphological generator} which delivers a TL SF
 for each TL LF, by suitably inflecting it. 
\item A \emph{reformatter} which de-encapsulates any format
  information.
\end{itemize}

\subsection{Morphological transducers}
There is one publicly available morphological tool for Indonesian, namely MorphInd \citep{larasati2011indonesian}. However, MorphInd is only designed for analysis, while we wanted the dictionary to be able to be used for both morphological analysis and generation. Moreover, there are rarely linguistic resources and tools available for Malaysian language. \citet{Baldwin06opensource} has developed a free/open-source lemmatiser for Malay, but this does not meet our need either since we wanted to include other important morphological information in our Malaysian transducer, such as parts of speech and affixes. Thus, we decided to build the morphological transducers from scratch.

Similar to most Apertium language pairs, the morphological transducers for both Indonesian and Malaysian are constructed using \emph{lttoolbox}, a toolbox for morphological analysis and generation that is available under free/open-source licence. The monolingual dictionary for each language is provided as XML-formatted entries, which is then compiled into a fast finite state transducers using the toolbox.

\subsubsection{Indonesian morphological transducer}
The words were added semi-automatically to the Indonesian morphological analyser based on frequency, with the most frequent words being added first. The frequency list was taken from a database dump of the Indonesian Wikipedia. For each word in the frequency list, we obtained its lemma and part of speech information from Kateglo\footnote{\url{http://kateglo.bahtera.org/}}, an online Indonesian dictionary with over 70,000 entries licensed under CC BY-SA 3.0\footnote{\url{http://creativecommons.org/licenses/by-sa/3.0/}}. Since we also wanted to include affix information in our analyser, we wrote a rule-based morpheme separator to decompose a given Indonesian surface form into their constituent morphemes, also by making use of the lemma information from Kateglo. Moreover, closed classes (e.g. pronouns, conjunctions) were added by hand.

\subsubsection{Malaysian morphological transducer}
A frequency list for Malaysian was also created based on a database dump of the Malaysian Wikipedia. Unlike Indonesian, we barely found a comprehensive Malaysian dictionary with adequate morphological information, such as lemma and part of speech. Hence, the Malaysian analyser was built using the two strategies below.

First, Malaysian words that also exist as an Indonesian word were assumed to share the same morphological information (i.e. the same lemma and part of speech), and added automatically to the analyser. Although this method may introduce a number of false friends (e.g. \emph{polisi} means `policy' in Malay but `police' in Indonesian), the benefit outweighs the risk since there is huge overlap in the lexicons of the two languages. Moreover, most of these false friends usually belong to the same part of speech.

Second, we also added Malaysian words which appear in our bilingual lexicon. Since every entry in a bilingual lexicon is a pair of words with the same meaning, we can assume that these words also belong to the same part of speech most of the time. Our approaches to building the bilingual dictionary are presented in the following section.

\subsection{Bilingual dictionary}

\begin{figure*}[htbp]
\begin{center}
\begin{small}
\begin{alltt}
<e r="RL"><p><l>\textbf{kemarin}<s n="adv"/></l>
             <r>\textbf{kelmarin}<s n="adv"/></r></p></e>
\end{alltt}
\end{small}
\caption{Example entries from the bilingual transfer lexicon. Indonesian is on the left, and Malay on the right. The attributes \texttt{LR} and \texttt{RL} indicate the direction restrictions.}
\label{fig:bidix}
\end{center}
%\vspace{-1em}
\end{figure*}

There is no freely available bilingual dictionary between Indonesian and Malaysian, so we had to build the dictionary from scratch. At the moment, the bilingual dictionary contains 12,142 entries, which was developed in several ways described below.

First, most of the entries were added using automatic word alignments. We created an Indonesian-Malaysian parallel corpus by translating many articles taken from Malaysian Wikipedia. The translation process is mostly automatic, with the help of existing Malaysian-Indonesian machine translation systems such as Google Translate.\footnote{\url{http://translate.google.com/}}

Next the Wikipedia corpus is tagged using our morphological analyser, and word alignments were created by running \texttt{Giza++} \citep{Och2003align} on the tagged corpus. We fed the probabilistic dictionary into the \texttt{ReTraTos} toolbox \citep{Caseli2006retratos}, which extracts both phrases and single-word translations from alignments, and converts them into Apertium translation entries. The \texttt{ReTraTos} method gave us about 12,000 translation entries, but also required a manual check due the amount of noise in the resulting data.

Finally, some entries were added manually, which included closed classes and words that frequently appeared in Wikipedia but were not yet added to the bilingual dictionary.

\subsection{Disambiguation}
The output from the morphological analysis is disambiguated using Apertium's statistical disambiguator module. The module implements a bigram part-of-speech tagger based on hidden Markov models (HMM). To improve the accuracy of our disambiguator, a Constraint Grammar \citep{Karlsson1990cg} could be used as a pre-disambiguator module before feeding the input to the HMM, which is left for future work.

\subsection{Lexical selection rules}
Given the closeness of the languages, lexical selection is not a large problem between Indonesian and Malaysian. However, a number of rules can be written for ambiguous words; for example, the Malaysian preposition \emph{daripada} `from (to explain the origin of something), than (comparison)' can be translated into Indonesian as either \emph{dari} `from' or \emph{daripada} `than (comparison)', depending on the surrounding context.

Another example is the copulas \emph{adalah} and \emph{ialah} (both meaning `be'), which exist in both Indonesian and Malaysian, but have a slightly different usage in each language. In Malaysian, \emph{adalah} is used before an adjective phrase or a prepositional phrase, and \emph{ialah} is used only before a noun phrase. In comparison to Indonesian, there are no strict rules governing the use of the two words, and their usage is more interchangeable.
\subsection{Transfer rules}
There are barely differences between the grammar of Indonesian and Malaysian, in that the structure of words, phrases, clauses, and sentences are almost exactly the same. That said, the lexical transfer between the two languages works by simple word substitution in most cases.

\begin{table*}[htbp]
\centering
\begin{tabular}{ll}
%\hline
%{\bf Stage} & {\bf Representation} \\
\toprule
{\bf (Malaysian) Input} & Cuaca kelmarin amatlah sejuk. \\ 
\midrule
%^Cuaca/Cuaca<n><sg>$ ^kelmarin/kelmarin<adv>$ ^amatlah/amatlah<adv>$ ^sejuk/sejuk<adj>$^./.<sent>$^./.<sent>$
{\bf Mor. analysis} & \^{}Cuaca/Cuaca\tag{<n>}\tag{<sg>\$} \^{}kelmarin/kelmarin\tag{<adv>\$}\\
~ & \^{}amatlah/amatlah\tag{<adv>\$} \^{}sejuk/sejuk\tag{<adj>\$}\\
~ & \^{}./.\tag{<sent>\$}\\
\midrule
% ^Cuaca<n><sg>$ ^kelmarin<adv>$ ^amatlah<adv>$ ^sejuk<adj>$^.<sent>$^.<sent>$
{\bf Mor. disambiguation}& \^{}Cuaca\tag{<n>}\tag{<sg>\$} \^{}kelmarin\tag{<adv>\$} \^{}amatlah\tag{<adv>\$}\\
~ & \^{}sejuk\tag{<adj>\$}\^{}.\tag{<sent>\$} \\
\midrule
% ^Cuaca<n><sg>$ ^kemarin<adv>$ ^amatlah<adv>$ ^dingin<adj>$^.<sent>$^.<sent>$
{\bf Transfer}& \^{}Cuaca\tag{<n>}\tag{<sg>\$} \^{}kemarin\tag{<adv>\$} \^{}amatlah\tag{<adv>\$}\\
~ & \^{}dingin\tag{<adj>\$}\^{}.\tag{<sent>\$} \\ 
\midrule
{\bf Mor. generation} & Cuaca kemarin amatlah dingin. \\
\bottomrule
\end{tabular}
 \caption{Translation process for the sentence \emph{Cuaca kelmarin amatlah sejuk.} `The weather yesterday is very cold'.}
\end{table*}

\section{Evaluation}
The system was evaluated in four ways. The first was the coverage\footnote{Here coverage is defined as \emph{na\"ive coverage}, that is for any given surface form at least one analysis is returned by our monolingual dictionaries} of the system. The second was the word error rate (WER) of the translation output for our test data set. The third was an analysis of the errors found by the second evaluation. Lastly, we did a comparative evaluation with an existing system.

\subsection{Coverage}
Lexical coverage of the system is calculated over the Indonesian and Malaysian Wikipedias, as shown in Table~\ref{table:coverage}. The database dump of the Indonesian Wikipedia\footnote{\url{http://id.wikipedia.org/}; \texttt{idwiki-20120429-pages-articles.xml.bz2}} was from the 29th April 2012, and that of Malaysian Wikipedia\footnote{\url{http://ms.wikipedia.org/}; \texttt{mswiki-20120428-pages-articles.xml.bz2}} from the 28th April 2012. Both database dumps were stripped of formatting.

\begin{table}[htbp]
  \begin{center}
  \begin{tabular}{ccc}
  \toprule
  Corpus & Tokens & Coverage\\
  \midrule
  Indonesian Wikipedia & 19,021,087 & 80.70\% \\
  \midrule
  Malaysian Wikipedia & 12,613,364 & 80.10\% \\
  \bottomrule
  \end{tabular}
  \caption{Na\"ive vocabulary coverage over Wikipedia.}
  \label{table:coverage}
  \end{center}
\end{table}

\subsection{Quantitative}
We tested our system on a 2,084 word text taken from various articles in Malaysian Wikipedia. The translation quality was measured using Word Error Rate (WER), a metric based on the Levenshtein distance \citep{levenshtein/1966}. We calculated the WER for each sentence using the \texttt{\small{apertium-eval-translator}}\footnote{\url{https://apertium.svn.sourceforge.net/svnroot/apertium/trunk/apertium-eval-translator/}} tool. The WER metric was preferred to other MT metric such as BLEU \citep{Papineni2002bleu} for assessing the postediting effort required.

For the Malaysian to Indonesian direction, the sentences were translated by the system, and then postedited by a native Indonesian speaker. For the Indonesian to Malaysian direction, we used the reference translation, as postedited by the native speaker and used it as a source of Indonesian to be translated to Malaysian, then the original Malaysian sentence was used as the reference translation.

\begin{table}[htbp]
  \begin{center}
  \begin{tabular}{ccccc}
  \toprule
  Corpus                 & Direction         & Tokens  & Unknown & WER  \\
  \midrule
  \multirow{2}{*}{Malaysian Wikipedia} & id$\rightarrow$ms & 2,079     & 211  & 14.43\% (83.89\%) \\
                          & ms$\rightarrow$id & 2,084     & 256  & 7.58\% (69.53\%)  \\
  \bottomrule
  \end{tabular}
    \caption{Word error rate over the Malaysian Wikipedia test data. Number in parentheses gives percentage of unknown words which were free rides.}
    \label{table:wer}
  \end{center}
\end{table}

We consider the WER of our system, as depicted in Table~\ref{table:wer} is quite acceptable for post-editing. Moreover, many of the unknown words are \emph{free rides}, which will not affect the final quality of the final translation.

\subsection{Qualitative}

\subsection{Comparative}
We compared our system to another MT system for Indonesian to Malaysian and Malaysian to Indonesian, Google Translate, a web-based statistical machine translation system. The evaluation was performed the same way: the test data was translated with Google Translate, then postedited.

\begin{table}[htbp]
  \begin{center}
  \begin{tabular}{clc}
  \toprule
  Dir.                 & System         & WER  \\
  \midrule
  \multirow{2}{*}{id$\rightarrow$ms} & \small{Apertium} & 14.43\% \\
                                     & \small{Google} & 13.90\% \\
  \midrule
  \multirow{2}{*}{ms$\rightarrow$id} & \small{Apertium} & 7.58\% \\
                                     & \small{Google} & 4.07\% \\
  \bottomrule
  \end{tabular}
    \caption{Accuracy comparison between the two systems.}
    \label{table:comp}
  \end{center}
\end{table}

We notice from Table~\ref{table:comp} that Google outperforms the Apertium system in both translation directions. For Malaysian to Indonesian, the error rate is reduced by almost a half. It is also interesting that both systems perform almost as worse for Indonesian to Malaysian, perhaps due to the fact that translating from Indonesian to Malaysian is more ambiguous then translating to Indonesian.

\section*{Conclusion and future work}

\section*{Acknowledgments}

\bibliographystyle{apa}

\bibliography{id-ms.coling2012}

\end{document}
